\documentstyle[jart12]{jarticle}
\pagestyle{plain}
\oddsidemargin=0.5cm
\evensidemargin=0.5cm
\textwidth=16cm
\textheight=24cm

\begin{document}
\hspace{80mm}様

\vspace{1cm}

前略、EusLispに関するお問い合わせを頂きありがとうございます。
テープをお送りします。テープには、ソース、sun4実行形式、sun3実行形式
が、QIC24フォーマットのtar形式で記録してあります。
次の手順で読み出して下さい。

\begin{verbatim}
% cd /usr/share/src; mkdir eus ; cd eus
% tar xvf /dev/nrst8 
% cd /usr/local ; mkdir eus; cd eus; mkdir bin; cd bin
% tar xvf /dev/nrst8
% mt -f /dev/rst8 rewind
\end{verbatim}

ディレクトリ名は、変更しても結構ですが、その場合は、
makefile, mkcommoneusの中のディレクトリ名を書き換えて下さい。
また、実行形式についても、次にようにして内部のディレクトリ名を
書き換えて下さい。

\begin{verbatim}
% cd /youreusdir/bin
% eus
EusLisp  6.866 created on Thu May 17 14:43:15 1990
eus$ *eusdir*
"/usr/local/eus/"
eus$ setq *eusdir* "/youreusdir/"
"/youreusdir/"
eus$ save "eus2"
t
eus$ ^D
% mv eus2 eus
\end{verbatim}

ディレクトリ内のリンクの張り方、ソースからのmakeの仕方、サンプルラン
についてはREADMEに解説してあります。
ドキュメントはdoc/latex, doc/ptroffに英語版、doc/jumanに日本語版があります。
ただし、どちらもまだ完全ではありません。

mailing-listで情報交換を図っています。
不明な点に関する質問や有用なプログラムの発表は、
euslisp@etl.go.jpに投稿されますようお願い申し上げます。



\vspace{1cm}

305 茨城県 つくば市 梅園 1-1-4

電子技術総合研究所

知能システム部 自律システム研究室

松井俊浩

0298-58-5977

0298-58-5971(fax)

\end{document}

