\section{System Functions}
\markright{\arabic{section}. System Functions}

\subsection{Memory Management}
The design of  memory management scheme affects much to the flexibility and
efficiency of object-oriented languages.
EusLisp allocates memory to any sort of objects
in a unified manner based on the {\em Fibonacci buddy method}.
In this method, each of large memory pools called chunks is split into
small cells which are unequally sized but aligned at Fibonacci numbers.
A memory chunk is a homogeneous data container for any types of
objects such as {\tt symbol, cons, string, float-vector}, etc.
as long as their sizes fit in the chunk.
A chunk has no special attributes,
like static, dynamic, relocatable, alternate, etc.
EusLisp's heap memory is the collection of chunks, and 
the heap can extend dynamically by getting new chunks from UNIX.
%In other words, there is no need to estimate the total amount of memory
%before EusLisp is invoked.
The expansion occurs either automatically on the fly
or on user's explicit demand by calling {\bf system:alloc} function.
When it is managed automatically, free memory size is kept
about 25\% of total heap size. % by default,
This ratio can be changed by setting a value between 0.1 and 0.9 to
the {\bf sys:*gc-margin*} parameter.

When all the heap memory is exhausted, mark-and-sweep type garbage collection
runs.
Cells accessible from the root (packages, classes and stacks) remain at the
same place where they were.
Other inaccessible cells are reclaimed and linked to the free-lists.
No copying or compactification occurs during GC.
When a garbage cell is reclaimed, its neighbor is examined
whether it is also free,
and they are merged together to form a larger cell if possible.
This merging, however, is sometimes meaningless,
since {\tt cons}, which is the most frequently called memory allocator,
requests the merged cell to be divided to the smallest cell.
Therefore, EusLisp allows to leave a particular amount of heap unmerged to speed up cons.
This ratio is determined by {\bf sys:*gc-merge*} parameter,
which is set to 0.3 by default.
With the larger {\bf sys:*gc-merge*}, the greater portion of  heap is left unmerged.
This improves the performance of consing,
since buddy-cell splitting rarely occurs when conses are requested.
This is also true for every allocation of relatively small cells,
like three dimensional float-vectors.

{\bf SYS:GC} invokes garbage collector explicitly, returning a list of two integers,
numbers of free words and total words (not bytes) allocated in the heap.
{\bf SYS:*GC-HOOK*} is a variable to hold a function that is called upon the
completion of a GC.  The hook function should receive two arguments
representing the sizes of the free heap and the total heap.

If "fatal error: stack overflow" is reported during  execution,
and you are convinced that the error is not caused by a infinite loop
or recursion, 
you can expand the size of the Lisp stack by {\bf sys:newstack}.
{\bf reset} should be performed before {\bf sys:newstack},
since it discards everything in the current stack such as
special bindings and clean-up forms of {\em unwind-protect}.
After a new stack is allocated, execution starts over from the point
of printing the opening message.
The default stack size is 65Kword.
The Lisp stack is different from the system stack.
The former is allocated in the heap, while the latter is allocated in
the stack segment by the operating system.
If you get "segmentation fault" error, it might be caused by the shortage
of the system stack.
You can increase the system stack size by the {\tt limit} csh command.

{\bf Sys:reclaim} and {\bf sys:reclaim-tree} function put cells occupied by objects
back to the memory manager, so that they can be reused later without
invoking garbage collection.
You must be assured that there remains no reference to the cell.

{\bf memory-report} and {\bf room} function display statistics on 
memory usage sorted by cell sizes and classes respectively.

%A user who tries to use {\bf address} and {\bf peek, poke} functions must be
%familiar with the structure of EusLisp data objects.
{\bf address} returns the byte address of the object and is useful
as a hash function when used with hash-table, since this address is
unique in the process.

{\bf Peek} and {\bf poke} are the functions to read/write data directly
from/to a memory location.
The type of access should be either of
{\bf :char, :byte, :short, :long, :integer, :float} and {\bf :double}.
For an instance, {\tt (SYS:PEEK (+ 2 (SYS:ADDRESS '(a b))) :short)}
returns class id of a cons cell, normally 1.

There are  several functions prefixed with 'list-all-'.
These functions returns the list of a system resource or environment,
and are useful for dynamic debugging.

\begin{refdesc}

\funcdesc{sys:gc}{}{
starts garbage collection, and returns a list of the numbers
of free words and total words allocated.}

\functiondescription{sys:*gc-hook*}{}{variable}{
Defines a function that is called upon the completion of a GC.}

\funcdesc{sys:gctime}{}{
returns a list of three integers: the count of gc invoked,
the time elapsed for marking cells (in 1/60 sec. unit),
and the time elapsed for reclamation (unmarking and merging).}

\funcdesc{sys:alloc}{size}{
allocates at least {\em size} words of memory in the heap,
and returns the number of words really allocated.}

\funcdesc{sys:newstack}{size}{
relinquishes the current stack,
and allocates a new stack of {\em size} words.}

\functiondescription{sys:*gc-merge*}{}{variable}{
is a memory management parameter.
{\bf *gc-merge*} is the ratio the ratio of heap memory  which is
left unmerged at GC.
This unmerged area will soon filled with smallest cells whose size is
the same as a cons.
The default value is 0.3.
The larger values, like 0.4, which specifies 40\% of free heap should be
unmerged, favors for consing but do harm to instantiating
bigger cells like float-vectors, edges, faces, etc.}

\functiondescription{sys:*gc-margin*}{}{variable}{
is a memory management parameter.
{\bf *gc-margin} determines the ratio of
free heap size versus the total heap.
Memory is acquired from UNIX so that the free space does not go 
below this ratio. The default value 0.25 means that 25\% of free space is
maintained at every GC.}

\funcdesc{sys:reclaim}{object}{
relinquishes {\em object} as a garbage.
It must be guaranteed that it is no longer referenced from any other objects.}

\funcdesc{sys:reclaim-tree}{object}{
reclaims all the objects except symbols traversable from {\em object}.}

\funcdesc{sys:btrace}{num}{
prints the back-trace information of {\em num} depth on the Lisp stack.}

\funcdesc{sys:memory-report}{\&optional strm}{
prints a table of memory usage report sorted by cell sizes
to the {\it strm} stream.}

\funcdesc{sys:room}{output-stream}{
outputs memory allocation information ordered by classes.}

\funcdesc{sys:address}{object}{
returns the address of {\it object} in the process memory space.
%This information is useful when it is sent to a user-coded C function.
}

\funcdesc{sys:peek}{[vector] address type}{
reads data at the memory location specified by {\em address}
and returns it as an integer.
{\em type} is one of {\bf :char, :byte, :short, :long, :integer, :float},
and {\bf :double}.
If no {\em vector} is given,
the address is taken in the unix's process space.
For example, since the a.out header is located at {\tt \#x2000} on SunOS4,
{\tt (sys:peek \#x2000 :short)} returns the magic number (usually {\tt \#o403}).
Solaris2 locates the ELF header at {\tt \#10000},
and {\tt (sys:peek \#x10000 :long)} returns {\tt \#xff454c46} whose
string representation is "ELF".

If {\em vector}, which can be a foreign-string,
is specified, address is recognized as an offset
from the vector's origin.
{\tt (sys:peek "123456" 2 :short)} returns short word
representation of "34", namely \#x3334 (13108).

Be careful about the address alignment:
reading short, integer, long, float, double word
at odd address may cause bus error by most CPU architectures.}

\funcdesc{sys:poke}{value [vector] address value-type}{
writes {\it value} at the location specified by {\it address}.
Special care should be taken since
you can write to anywhere in the process memory space.
Writing to outside the process space surely causes segmentation fault.
Writing short, integer, long, float, double word
at odd address causes bus error.}

\funcdesc{sys:list-all-chunks}{}{
list up all allocated heap chunks. Not useful for other than the implementor.}

\funcdesc{sys:object-size}{obj}{
counts the number of cells and words accessible from {\em obj}.
All the objects reference-able from obj are traversed,
and a list of three numbers is returned:
the number of cells, 
the number of words logically allocated to these objects
(i.e. accessible from users),
and the number of words
physically allocated including headers and extra slots for memory management.
Traversing stops at symbols, i.e. objects referenced from a symbol such as
property-list or print-name string are not counted.}

\end{refdesc}

\newpage
\subsection{Unix System Calls}

EusLisp assorts functions which directly correspond to
the system calls and library functions of UNIX operating system.
For further detail of these functions, consult UNIX system interface
reference (2).
These low-level functions defined in {\tt *unix-package*} are sometimes dangerous.
Use higher level functions defined in other packages if possible.
For example, use IPC facilities described in the section \ref{IPC}
instead of {\tt unix:socket, unix:bind, unix:connect}, and so on.

\subsubsection{Times}
\begin{refdesc}
\funcdesc{unix:ptimes}{}{
a list of five elements, elapsed time, system time, user time,
subprocess's system time, subprocess's user time, is returned.
Unit is always one sixtieth second.
This function is obsolete and use of {\bf unix:getrusage} is recommended.}

\funcdesc{unix:runtime}{}{
Sum of the process's system and user time is returned.
Unit is 1/60 second.}

\funcdesc{unix:localtime}{}{
Current time and date is returned in an integer vector.
Elements are second, minute, hour, day-of-a-month, year, weekday.}

\funcdesc{unix:asctime}{ tm\_intvector}{
Converts localtime represented with an integer-vector into a string notation.\\
{\tt (unix:asctime (unix:localtime))} returns a string representation of
the current real time.}

\end{refdesc}

\subsubsection{Process}
\begin{refdesc}
\funcdesc{unix:getpid}{}{
returns the process id (16bit integer) of this process.}

\funcdesc{unix:getppid}{}{
returns the process id of the parent process.}

\funcdesc{unix:getpgrp}{integer}{
returns the process group id.}

\funcdesc{unix:setpgrp}{integer}{
sets a new process group id.}

\funcdesc{unix:getuid}{}
{gets user id of this process.}

\funcdesc{unix:geteuid}{}{
returns the effective user id of this process.}

\funcdesc{unix:getgid}{}{
returns the group id of this process.}

\funcdesc{unix:getegid}{}{
returns the effective group id of this process.}

\funcdesc{unix:setuid}{integer}{
sets effective user id of this process.}

\funcdesc{unix:setgid}{integer}{
sets the effective group id of this process.}


\funcdesc{unix:fork}{}{
creates another EusLisp. 0 is returned to the subprocess and the pid of
the forked process is returned to the parent process.
Use {\bf system:piped-fork} described in section \ref{UnixProcess}
to make a process connected via pipes.}

\funcdesc{unix:vfork}{}{
forks another EusLisp, and suspends the parent process from execution
until the new EusLisp process terminates.}

\funcdesc{unix:exec}{path}{
replaces executing EusLisp with another program.}

\funcdesc{unix:wait}{}{
waits for the completion of one of subprocesses.}

\funcdesc{unix:exit}{code}{
terminates execution and returns {\em code} as its completion status.
Zero means normal termination.}

\functiondescription{sys:*exit-hook*}{}{variable}{
Defines a function that is called just before the process is exited.}

\funcdesc{unix:getpriority}{which who}{
returns the highest priority (nice value) enjoyed by this process.
{\em Which} is one of 0(process), 1(process-group) or 2(user).}

\funcdesc{unix:setpriority}{which who priority}{
sets priority 
of the resource determined by {\em which} and {\em who}.
{\em which} is one of 0(process), 1(process-group) or 2(user).
{\em who}  is  interpreted relative to {\em which}
(a process identifier
for $which=0$, process group  identifier  for  $which=1$,
and  a  user ID for $which=2$.
A zero value of who denotes
the current process, process group, or user.
To lower the priority  (nice value) of your EusLisp process,
{\tt (unix:setpriority 0 0 10)} will sets the nice value to 10.
Bigger nice value makes your process get less favored.}

\funcdesc{unix:getrusage}{who}{
returns list of system resource usage information about {\em who} process.
Elements are ordered as follows:
More comprehensive display is obtained by {\bf lisp:rusage}.}
{\small
\begin{verbatim}
    float ru_utime (sec.) /* user time used */
    float ru_stime (sec.) /* system time used */
    int  ru_maxrss;       /* maximum resident set size */
    int  ru_ixrss;        /* currently 0 */
    int  ru_idrss;        /* integral resident set size */
    int  ru_isrss;        /* currently 0 */
    int  ru_minflt;       /* page faults without physical I/O */
    int  ru_majflt;       /* page faults with physical I/O */
    int  ru_nswap;        /* number of swaps */
    int  ru_inblock;      /* block input operations */
    int  ru_oublock;      /* block output operations */
    int  ru_msgsnd;       /* messages sent */
    int  ru_msgrcv;       /* messages received */
    int  ru_nsignals;     /* signals received */
    int  ru_nvcsw;        /* voluntary context switches */
    int  ru_nivcsw;       /* involuntary context switches */
\end{verbatim}
}

\funcdesc{unix:system}{[command]}{
executes {\em command} in a sub shell.
{\em command} must be recognizable by Bourn-shell.}

\funcdesc{unix:getenv}{env-var}{
gets the value for the environment variable {\em env-var}.}

\funcdesc{unix:putenv}{env}{
adds {\em env} in the process's environment variable list.
{\em env} is a string which equates var to value
like {\tt "VARIABLE=value"}.}

\funcdesc{unix:sleep}{time}{
suspends execution of this process for {\em time} seconds.}

\funcdesc{unix:usleep}{time}{
suspends execution of this process for {\em time} micro-seconds
({\bf u} represents micro).
{\bf Usleep} is not available on Solaris2 or other Sys5 based systems.}

\end{refdesc}

\subsubsection{File Systems and I/O}
\begin{refdesc}

\funcdesc{unix:uread}{stream [buffer] [size]}{
reads {\em size} bytes from {\em stream}.
{\em stream} may either be a stream object or an integer representing fd.
If {\em buffer} is given, the input is stored there.
Otherwise, input goes to the buffer-string in  {\em stream}.
Therefore, if {\em stream} is fd, {\em buffer} must be given.
{\bf unix:uread} never allocates a new string buffer.
{\bf unix:uread} returns the byte count actually read.}

\funcdesc{unix:write}{stream string \&optional size}{
writes {\em size} bytes of {\em string} to {\em stream}.
If {\em size} is omitted, the full length of {\em string} is output.}

\fundesc{unix:fcntl}{stream command argument}
\fundesc{unix:ioctl}{stream command buffer}
\fundesc{unix:ioctl\_}{stream command1 command2}
\fundesc{unix:ioctl\_R}{stream command1 command2 buffer [size]}
\fundesc{unix:ioctl\_W}{stream command1 command2 buffer [size]}
\fundesc{unix:ioctl\_WR}{stream command1 command2 buffer [size]}
\funcdesc{unix:close}{fd}{
close a file specifying its file descriptor {\em fd}.}
\funcdesc{unix:dup}{fd}{
returns the duplicated file descriptor for {\em fd}.}
\funcdesc{unix:pipe}{}{creates a pipe.
An io-stream for this pipe is returned.}
\funcdesc{unix:lseek}{stream position [whence 0]}{
sets the file pointer for {\em stream} at {\em position} counted
from {\em whence}.}

\funcdesc{unix:link}{path1 path2}{makes a hard link.}

\funcdesc{unix:unlink}{path}{removes a hard link to the file specified by {\em path}.
If no reference to the file lefts, it is deleted.}

\funcdesc{unix:mknod}{path mode}{makes inode in a file system.
{\em path} must be a string, not a pathname object.}
\funcdesc{unix:access}{path mode}{
checks the access rights to {\em path}.}
\funcdesc{unix:stat}{path}{
gets inode information of {\em path} and
returns a list of integers  described below.}
{\small
\begin{verbatim}
  st_ctime	; file last status change time
  st_mtime	; file last modify time
  st_atime	; file last access time
  st_size	; total size of file, in bytes
  st_gid	; group ID of owne
  st_uid	; user ID of owner
  st_nlink	; number of hard links to the file
  st_rdev	; the device identifier (special files only)
  st_dev	; device file resides on
  st_ino	; the file serial number
  st_mode	; file mode
\end{verbatim}
}
\funcdesc{unix:chdir}{path}{changes the current working directory to {\em path}.}
\funcdesc{unix:getwd}{}{gets current working directory.}

\funcdesc{unix:chmod}{path integer}{
changes access mode (permission) for {\em path}.}
\funcdesc{unix:chown}{path integer}{
changes the owner of the file {\em path}.}

\funcdesc{unix:isatty}{(stream $|$ fd)}{
returns T if {\em stream} is connected to a tty-type
device (a serial port or a pseudo tty) .}

\funcdesc{unix:msgget}{key mode}{
creates or allocates a message queue which is addressed by {\em key}.}
\fundesc{unix:msgsnd}{qid buf [mtype [flag]]}
\fundesc{unix:msgrcv}{qid buf [size [mtype [flag]]]}

\funcdesc{unix:socket}{domain type \&optional proto}{creates a socket
whose name is defined in {\it domain} and whose abstract type is {\em type}.
{\em type} should be one of 1 (SOCK\_STREAM),
2 (SOCK\_DGRAM), 3 (SOCK\_RAW), 4 (SOCK\_RDM) and 5 (SOCK\_SEQPACKET).}
\funcdesc{unix:bind}{socket name}{
associates {\em name} to {\em socket}.
{\em name} should be a unix path-name if the socket is defined in unix-domain.}
\funcdesc{unix:connect}{socket addr}{
connects {\em socket} to another socket specified by {\em addr}.}
\funcdesc{unix:listen}{socket \&optional backlog}{
begins to accept connection request on {\em socket}.
{\em backlog} specifies the length of the queue waiting for the establishment
of connection.}
\funcdesc{unix:accept}{socket}{accepts the connection request on {\em socket}
and returns a file-descriptor on which messages can be exchanged
bidirectionally.}
\funcdesc{unix:recvfrom}{socket \&optional mesg from flag}{
receives a datagram message from {\em socket}.
The socket must be assigned a name by {\bf unix:bind}.
{\em mesg} is a string in which the incoming message will be stored.
If {\em mesg} is given, {\bf recvfrom} returns the number of bytes received.
If it is omitted, a new string is created for the storage of the message
and returned.}
\funcdesc{unix:sendto}{socket addr mesg \&optional len flag}{
sends a datagram message to another socket specified by {\em addr}.
{\em Socket} must be a datagram-type socket which has no name assigned.
{\em Mesg} is a string to be sent and {\em len} is the length of
the message counting from the beginning of the string.
If omitted, whole string is sent.
}

\funcdesc{unix:getservbyname}{servicename}{
returns the service number (integer) for {\em servicename} registered
in {\tt /etc/services} or in NIS database.}

\funcdesc{unix:gethostbyname}{hostname}{
returns the list  of ip address of {\em hostname} and its address type
(currently always AF\_INET==2).}

\funcdesc{unix:syserrlist}{errno}{
returns a string describing the error information for the error code
{\em errno}.}

\end{refdesc}


\subsubsection{Signals}

\begin{refdesc}
\funcdesc{unix:signal}{signal func [option]}{
installs the signal handler {\em func} for {\em signal}.
In BSD4.2 systems, signals caught during system call processing
cause the system call to be retried.
This means that if the process is issuing a read system call,
signals are ignored.
If {\em option=2} is specified,
signals are handled in the system-5 manner,
which causes the system call to fail. }

\funcdesc{unix:kill}{pid signal}{
sends a signal to a process named by {\em pid}.}

\funcdesc{unix:pause}{}{
suspends execution of this process until a signal arrives.}

\funcdesc{unix:alarm}{time}{
sends an alarm clock signal (SIGALRM 14) after {\em time} seconds.
Calling {\bf unix:alarm} with {\em time}=0 resets the alarm clock.}

\funcdesc{unix:ualarm}{time}{
same as {\bf unix:alarm} except that the unit of {\em time} is micro seconds.
{\bf ualarm} is not available on Solaris2 or on other Sys5 based systems.}

\funcdesc{unix:getitimer}{timer}{
One Unix process is attached with three interval timers, i.e.,
a real-time timer that decrements as the real time passes,
a virtual-timer that decrements as the process executes in the user space,
and a prof-timer that decrements as the kernel executes on behalf of
the user process.
{\em timer} is either  0 ({\tt ITIMER\_REAL}), 1 ({\tt ITIMER\_VIRTUAL}),
or 2({\tt ITIMER\_PROF}).
A list of two elements is returned,  the value of the timer in second
and the interval. Both are floating-point numbers.}

\funcdesc{unix:setitimer}{timer value interval}{
sets {\em value} and {\em interval} in {\em timer}.
{\em timer} is eiterh  0 ({\tt ITIMER\_REAL}), 1 ({\tt ITIMER\_VIRTUAL}),
or 2({\tt ITIMER\_PROF}).
{\tt ITIMER\_REAL} delivers SIGALRM when {\em value} expires.
{\tt ITIMER\_VIRTUAL} delivers SIGVTALRM, and 
{\tt ITIMER\_PROF} delivers SIGPROF.}

\funcdesc{unix:select}{inlist outlist exceptlist timeout}{
{\em inlist, outlist} and {\em exceptlist} are bitvectors
indicating file descriptors whose I/O events should be tested.
For example, if {\em inlist}=\#b0110, {\em outlist}=\#b100,
and {\em exceptlist}=NIL,
then whether it is possible to read on fd=1 or 2, or to write on fd=2
is tested.
{\em Timeout} specifies seconds for which {\bf select} is allowed to
wait.
Immediately after
incoming data appear on one of the ports specified in {\em inlist},
or writing become available on one of the ports specified in {\em outlist},
or exceptional condition arises in one of the ports specified in
{\em exceptlist},
{\bf select} returns the number of ports that are available for I/O
operation, setting ones for the possible port s
in each of {\em inlist, outlist and exceptlist}.}

\funcdesc{unix:select-read-fd}{read-fdset timeout}{
I/O selection is usually meaningful only for input operation.
{\bf unix:select-read-fd} is a short-hand for
{\tt select fdset nil nil timeout}.
{\em Read-fdset} is not a bit-vector,
 but an integer that specifies the reading fd set.}

\end{refdesc}

\subsubsection{Multithread}
There is no way to create bound threads. Therefore only one signal stack
and one interval timer are available in a EusLisp process.
On Solaris2, the main top-level runs in a separated thread.

\begin{refdesc}
\funcdesc{unix:thr-self}{}{returns the id (integer) of the thread currently
running.}

\funcdesc{unix:thr-getprio}{id}{returns the execution priority of the thread
specified by {\em id}.}
\funcdesc{unix:thr-setprio}{id newprio}{sets the execution priority of
the thread specified by {\em id} to {\em newprio}.
The smaller numerical value of {\em newprio} means the higher priority.
In other words,
a thread with a numerically greater {\em newprio} gets less access to CPU.
Users cannot raise the execution priority higher than
the process's nice value, which is usually 0.}

\funcdesc{unix:thr-getconcurrency}{}{
returns the concurrency value (integer) which represents the number
of threads that can run concurrently.}
\funcdesc{unix:thr-setconcurrency}{concurrency}{
The concurrency value is the number of LWP in the process.
If the concurrency is 1, which is the default, many threads you created
are assigned to one LWP in turn even though all of them are runnable.
If the program is running on a multi-processor machine and you want
to utilize more than one CPU at the same time,
you should set a value bigger than one to {\em concurrency}.
Note that a big concurrency value let the operating system consume
more resource. Usually {\em concurrency} should be smaller than or
equal to the number of processors.}

\funcdesc{unix:thr-create}{func arg-list \&optional (size 64*1024)}{
creates a new thread with {\em size} words of Lisp stack and
{\em size} bytes of C stack,
and let it apply {\em func} to {\em arg-list}.
The thread cannot return any results to the caller.
Use of this function is discouraged.}

\end{refdesc}

\subsubsection{Low-Level Memory Management}
\begin{refdesc}
\funcdesc{unix:malloc}{integer}{
allocates memory outside EusLisp memory space.}
\funcdesc{unix:free}{integer}{
deallocates a  memory block allocated by {\em unix:malloc}.}
\fundesc{unix:valloc}{integer}
\fundesc{unix:mmap}{address length protection share stream offset}
\fundesc{unix:munmap}{address length}
\fundesc{unix:vadvise}{integer}

\end{refdesc}

\subsubsection{IOCTL}
Although Unix controls terminal device by a set of commands (second argument) to 
{\tt ioctl}, EusLisp provides them in the forms of function to eliminate
to reference the include files and or'ing argument with the command codes.
For the detail, refer to the {\em termio} manual pages of Unix.

There are two sets of terminal io-controls: TIOC* and TC*.
Be careful about the availability of these functions on your operating system.
Basically, BSD supports TIOC* io-controls and Sys5 supports TC*.
\begin{description}
\item{SunOS 4.1} Both TIOC* and TC*
\item{Solaris2} only TC*
\item{mips, ultrix?} only TIOC*
\end{description}

\begin{refdesc}

\funcdesc{unix:tiocgetp}{stream [sgttybuf]}{gets parameters.}
\funcdesc{unix:tiocsetp}{stream sgttybuf}{sets parameters.}
\fundesc{unix:tiocsetn}{stream [sgttybuf]}
\fundesc{unix:tiocgetd}{stream [sgttybuf]}
\funcdesc{unix:tiocflush}{stream}{flushes output buffer}
\funcdesc{unix:tiocgpgrp}{stream integer}{gets process group id.}
\funcdesc{unix:tiocspgrp}{stream integer}{sets process group id.}
\fundesc{unix:tiocoutq}{stream integer}
\fundesc{unix:fionread}{stream integer}
\fundesc{unix:tiocsetc}{stream buf}
\fundesc{unix:tioclbis}{stream buf}
\fundesc{unix:tioclbic}{stream buf}
\fundesc{unix:tioclset}{stream buf}
\fundesc{unix:tioclget}{stream buf}

\funcdesc{unix:tcseta}{stream buffer}{sets terminal parameters immediately.}
\funcdesc{unix:tcsets}{stream buffer}{sets terminal parameters.}
\funcdesc{unix:tcsetsw}{stream buffer}{sets terminal parameters after all
characters queued for output have been transmitted.}
\funcdesc{unix:tcsetsf}{stream buffer}{sets terminal parameters after all
characters queued for output have been transmitted and all characters
queued for input are discarded.}
\fundesc{unix:tiocsetc}{stream buffer}
\fundesc{unix:tcsetaf}{stream buffer}
\fundesc{unix:tcsetaw}{stream buffer}
\fundesc{unix:tcgeta}{stream buffer}
\fundesc{unix:tcgets}{stream buffer}
\fundesc{unix:tcgetattr}{stream buffer}
\fundesc{unix:tcsetattr}{stream buffer}
\end{refdesc}

\subsubsection{Keyed Indexed Files}
Recent Unix provides with the {\em dbm} or {\em ndbm} library
for the management of keyed index files.
Making use of this library, you can build a data base  that is composed
of many pairs  of key and datum association.
Following functions are defined in clib/ndbm.c.
On Sun, it should be compiled by {\tt cc -c -Dsun4 -Bstatic},
and loaded into EusLisp by {\tt (load "clib/ndbm" :ld-option "-lc")}.

\begin{refdesc}
\funcdesc{dbm-open}{dbname mode flag}{
{\bf Dbm-open} must be called first to create a data base file,
and to begin read/write operations to the data base.
{\em Dbname} is the name of the data base.
Actually, ndbm manager creates two files which have suffixes {\tt ".pag"}
and {\tt ".dir"}.
{\em Mode} specifies file-open mode; 0 for read-only access, 1 for write-only,
and 2 for read-write; also \#x200 should be {\em or}ed when you create
the file  at the first time.
{\em Flag} gives access permission that is changed by chmod.
\#o666 or \#o664 is good for {\em flag}.
{\bf Dbm-open} returns an integer that identifies the data base in the process.
This value is used by other dbm functions to identify the data base.
In other words, you can open several data bases at the same time.}

\funcdesc{dbm-store}{db key datum mode}{
stores {\em key-datum} association in {\em db}.
{\em Db} is an integer to identify the data base.
{\em Key and datum} are strings.
{\em Mode} is 0 (insert) or 1 (replace).}

\funcdesc{dbm-fetch}{db key}{
retrieves datum that is associated with {\em key} in {\em db}.}

\end{refdesc}

\newpage
\subsection{\label{UnixProcess}Unix Processes}

In order to launch unix commands from EusLisp,
use the {\bf unix:system} function.
{\bf Piped-fork} creates a subprocess whose standard input and standard output
are connected to EusLisp's bidirectional stream through pipes.
{\bf Piped-fork} returns the stream.
Following is a function to count the number of lines contained in a file
by using {\tt "wc"}.
\begin{verbatim}
(defun count-lines (file) (read (piped-fork "wc" file)))
\end{verbatim}

The next example creates eus process on another workstation identified by "etlic0"
and provides a port for distributed computation.

\begin{verbatim}
(setq ic0eus (piped-fork "rsh" "etlic0" "eus"))
(format ic0eus "(list 1 2 3)~%")
(read ic0eus) --> (1 2 3)
\end{verbatim}

For source code editing, you can call {\tt ez} from the EusLisp.
The screen editor ez communicates with EusLisp through message-queues.
If you have an ez process already running in parallel with the EusLisp,
{\bf ez} restarts ez and it gains the terminal control. 
By issuing esc-P or esc-M commands in ez,
texts are sent back and evaluated by EusLisp.
This is useful for the debugging since entire file does not need to be
loaded when you add a little modification to the file.
Similar function is available on emacs by M-X run-lisp command.

\begin{refdesc}
\funcdesc{cd}{\&optional (dir (unix:getenv "HOME"))}{
changes the current working directory.}

\funcdesc{ez}{\&optional key}{
enters display editor ez, and reads Lisp forms from it, and evaluates
them.}

\funcdesc{piped-fork}{\&optional (exec) \&rest args}{
forks a process, and makes a two-way stream between the current EusLisp and the
subprocess. Exec is the file name of a unix command and args are arguments
to the command.  If exec (string) includes one or more space, it is
assumed a shell command, and executed by /bin/sh calling the unix:system function.
If no exec is given, another euslisp is created as the subprocess.}

\funcdesc{xfork}{exec \&key (stdin *standard-input*) (stdout *standard-output*) \\
(stderr *error-output*) (args nil)}{
forks a process, replaces its stdin, stdout, and stderr streams to specified ones,
and exec's "exec" with the args arguments.
piped-fork is roughly equivalent to
\verb~ (xfork exec :stdin (unix:pipe) :stdout (unix:pipe))~
Though xfork returns an io-stream to stdin and stdout with their
directions reversed,
it is not always useful unless they are pipes. 
The name of this function, xfork (cross-fork), comes from this reversed io-stream,
namely, the io-stream's input comes from the stdout of the subprocess and the output
comes from the stdin. }


\funcdesc{rusage}{}{prints resource usage of this process.}
\end{refdesc}

\newpage

\subsection{Adding Lisp Functions Coded in C}

Programs that heavily refer to C include files or frequently access arrays
perform better or are more clearly described
if written in C or other languages rather than in EusLisp.
EusLisp provides the way to link programs coded in C.

If you want to define EusLisp function written in C,
each EusLisp-callable C-function must be coded to accept three arguments:
the context pointer, the number of arguments and the pointer to the Lisp
argument block.
These arguments must be named as {\tt ctx, n} and {\tt argv},
since the macros in {\tt c/eus.h} assume these names.
The C program must include {\tt *eusdir*/c/eus.h}.
The programmer should be familiar with the types and macros
described there.
The entry function should be named by the basename of the source file.

A sample code for C function AVERAGE which computes the arithmetic
average of arbitrary number of floats is shown below.
In this example, you can see how to get float values from arguments,
how to make the pointer of a float,
how to set a pointer in the special variable AVERAGE,
and how to define a function and a symbol in the entry function {\tt ave}.
Compile this program by {\tt 'cc -c -Dsun4 -DSolaris2 -K pic'}.
{\tt -Dsun4} and {\tt -DSolaris2} are needed
to chose proper definitions in {\tt c/eus.h}.
{\tt -K pic} is needed to let the c compiler generate position independent
code necessary for the loadable shared object. 
Then the resulted '.o' file can be loaded into EusLisp.
More complete examples can be found in {\tt *eusdir*/clib/*.c},
which are defined and loaded in the same manner described here.

\begin{verbatim}
/* ave.c */
/* (average &rest numbers) */
#include "/usr/local/eus/c/eus.h"
static pointer AVESYM;
pointer AVERAGE(ctx,n,argv)
context *ctx;
int n;
pointer argv[];
{ register int i;
  float sum=0.0, a, av;
  pointer result;
  numunion nu;
  for (i=0; i<n; i++) {
    a=ckfltval(argv[i]);
    sum += a;} /*get floating value from args*/
  av=sum/n;
  result=makeflt(av);
  AVESYM->c.sym.speval=result;  /*kindly set the result in symbol*/
  return(result);}

ave(ctx,n,argv)
context *ctx;
int n;
pointer argv[];
{ char *p;
  p="AVERAGE";
  defun(ctx,p,argv[0],AVERAGE);
  AVESYM=intern(ctx,p,strlen(p),userpkg); /* make a new symbol*/
  }     

\end{verbatim}

\subsection{Foreign Language Interface}

Functions written in C without concern about linking with EusLisp
can be loaded onto EusLisp, too.
These functions are called foreign functions.
Such programs are loaded by
{\bf load-foreign} macro which returns an instance of {\bf foreign-module}.
External symbol definitions in the object file is registered
in the module object.
{\bf Defforeign} is used to make entries to  C functions
to be called from EusLisp.
{\bf Defun-c-callable} defines lisp functions callable from C.
C-callable functions have special code piece called {\em pod-code}
for converting parameters and transferring control to the corresponding
EusLisp function.
{\bf Pod-address} returns the address of this code piece which
should be informed to C functions.

Here is an example of C program and its interface functions to EusLisp.
\begin{verbatim}
/* C program named cfunc.c*/

static int (*g)();	/* variable to store Lisp function entry */

double sync(x)
double x;
{ extern double sin();
  return(sin(x)/x);}

char *upperstring(s)
char *s;
{ char *ss=s;
  while (*s) { if (islower(*s)) *s=toupper(*s); s++;}
  return(ss);}

int setlfunc(f)      /* remember the argument in g just to see */
int (*f)();          /* how Lisp function can be called from C */
{ g=f;}

int callfunc(x)      /* apply the Lisp function saved in g to the arg.*/
int x;
{ return((*g)(x));}

;;;; Example program for EusLisp's foreign language interface
;;;; make foreign-module
(setq m (load-foreign "cfunc.o"))

;; define foreign functions so that they can be callable from lisp
(defforeign sync m "sync" (:float) :float)
(defforeign toupper m "upperstring" (:string) :string)
(defforeign setlfunc m "setlfunc" (:integer) :integer)
(defforeign callfunc m "callfunc" (:integer) :integer)

;; call them
(sync 1.0)	--> 0.841471
(print (toupper "abc123"))  --> "ABC123"

;; define a test function which is callable from C.
(defun-c-callable TEST ((a :integer)) :integer
      (format t "TEST is called, arg=~s~%" a)
      (* a a))    ;; return the square of the arg
;;  call it from C
;;setlfunc remembers the entry address of Lisp TEST function.
(setlfunc (pod-address (intern "TEST")))
(callfunc 12)  --> TEST is called, arg=12  144
\end{verbatim}
  
Data representations in EusLisp are converted to those of C in the following
manners:
EusLisp's 30-bits integer (including character)
is sign-extended and passed to a C function via stack.
30-bit float is extended to double and passed via stack.
As for string, integer-vector and float-vector,
only the address of the first element is passed on the stack,
and the entire array remains uncopied.
The string can either be a normal string or a foreign-string.
A string may contain null codes, though it is guaranteed
that the string also has a null code at the end.
EusLisp does not know how to pass arrays of more than one dimension.
Every array of more than one dimension has correspoiding one dimensional
vector that holds the entire elements linearly.
This vector is obtained by the {\bf array-entity} macro.
Also, note that a two-dimensional matrix should be transposed
if it is sent to the FORTRAN subroutines, since rows and columns
are ordered oppositely in FORTRAN.

Since EusLisp's representation of floating-point numbers is always single
precision, conversion is required when you pass a vector of double precision
floating point numbers.
For this purpose, the conversion functions,
{\bf double2float} and {\bf float2double} are provided
by {\tt clib/double.c}.
For an instance, if you have a 3x3 float-matrix and want to pass it to a C
function named cfun as a matrix of double, use the following forms.
\begin{verbatim}
     (setq mat (make-matrix 3 3))
     (cfun (float2double (array-entity mat)))
\end{verbatim}

Struct in C can be defined by the {\bf defcstruct} macro.
{\bf Defcstruct} accepts struct-name followed by 
field definition forms.

\begin{verbatim}
     (defcstruct <struct-name>
        {(<field> <type> [*] [size])}*)
\end{verbatim}
For example, following struct definition is represented by the next defcstruct.
\begin{verbatim}
     /* C definition */
     struct example {
        char  a[2];
        short b;
        long  *c;
        float *d[2];};

     /* equivalent EusLisp definition */
     (defcstruct example
        (a :char 2)
        (b :short)
        (c :long *)
        (d :float * 2))
\end{verbatim}

\begin{refdesc}

\macrodesc{load-foreign}{objfile \&key symbol-input symbol-output (symbol-file objfile) ld-option)}
{loads an object module written in languages other than EusLisp.
In Solaris2, {\bf load-foreign} just calls {\bf load} with a null string
as its {\em :entry} parameter.
A compiled-code object is returned.
This result is necessary to make entries to the functions in the module by
{\bf defforeign} called later on.
Libraries can be specified in {\em ld-option}.
However, the symbols defined in the libraries cannot be captured
in the default symbol-output file.
In order to allow EusLisp to call functions defined in the libraries,
{\em symbol-output} and {\em symbol-file} must be given explicitly.
(These arguments are not needed if you are not going to call the library
functions directly from EusLisp, i.e. if you are  referring them only from
functions in {\em objfile}).
{\bf Load-foreign} links {\em objfile} with libraries specified and global
symbols in EusLisp which is in core, and writes the linked object in
{\em symbol-output}.
Then, symbols in {\em symbol-file} are searched and listed in the
foreign-module. Since {\em symbol-file} is defaulted to be {\em objfile},
only the symbols defined in {\em objfile} are  recognized if {\em symbol-file}
is not given. To find all the global entries both in {\em objfile} and
libraries, the linked (merged) symbol table resulted from the first link
process of load-foreign must be examined.
For this reason, an identical file name must be given both to {\em symbol-output}
and to {\em symbol-file}.

As shown below, the intermediate symbol file can be removed
by {\bf unix:unlink}.
However, if you are loading more than one foreign modules both of which 
refer to the same library, and if you want to avoid loading
the library duplicatedly, you have to use {\em symbol-input} argument.
Suppose you have loaded all the functions in "linpack.a" in the above
example and you are going to load another file "linapp.o" that calls functions
in "linpack.a".
The following call of load-foreign should be issued before you
unlink "euslinpack".
{\tt (load-foreign "linapp.o" :symbol-input "euslinpack")}
See {\tt *eusdir*/llib/linpack.l} for more complete examples of
{\bf load-foreign} and {\bf defforeign}.
}
\begin{quote}
\begin{verbatim}
(setq linpack-module
    (load-foreign "/usr/local/eus/clib/linpackref.o"
        :ld-option  "-L/usr/local/lib -llinpack -lF77 -lm -lc"
        :symbol-output "euslinpack"
        :symbol-file "euslinpack"
        ))
(unix:unlink "euslinpack")
\end{verbatim}
\end{quote}

\macrodesc{defforeign}{funcname module cname paramspec resulttype}{
makes a function entry in a foreign language module.
{\em funcname} is a symbol to be created in EusLisp.
{\em module} is  a compiled-code object returned by {\bf load-foreign}.
{\em cname} is the name of the C-function defined in the foreign program.
It is a string like "\_myfunc". 
%Note that Sun's cc unconditionally prefixes the function name with  "\_".
{\em paramspec} is a list of parameter type specifications
which is used for the data type conversion and coercion when arguments are
passed from EusLisp to the C function.
{\em Paramspec} can be NIL if no data conversion or type check is required.
One of {\bf :integer, :float , :string}, or {\em (:string n)} must be
given to {\em resulttype}.
{\bf :Integer} means that the c function returns either char, short or
int (long).
{\bf :Float} should be specified both for float and double.
{\bf :String} means the C function returns a pointer to a string,
and EusLisp should add a long-word header to the  string to accomodate it
as a EusLisp string.
The length of the string is found by {\em strlen}.
Note that the writing a header just before the string may cause
a disastrous result.
On the other hand, {\bf (:string n)} is safer but slower because
a EusLisp string of length {\bf n} is newly created and the contents of
C string is copied there.
{\tt (:string 4)} should be used for a C function that returns a pointer
to an integer.  The resulted integer value of the result can be obtained by
{\tt (sys:peek result :long)}, where result is a variable set to the
result of the C function.
You may also specify {\tt (:foreign-string [n])} for C functions
that return a string or a struct.
The result is a foreign-string whose content is held somewhere outside
EusLisp control.
If the result string is null-terminated and the length of the string
is known by strlen, you don't need to specify the length [n].
However, if the result contains null codes, which is usual for structs,
the length of the foreign-string should be explicitly given.
Whether you should use (:string n) or (:foreign-string n) is not
only the matter of speed, but the matter of structure sharing.
The difference is whether the result is copied or not.

Fortran users should note that every argument to a Fortran function or a
subroutine is passed by call-by-reference.
Therefore, even a simple integer or float type argument must be  put
in a integer-vector or a float-vector before it is passed to Fortran.}

\macrodesc{defun-c-callable}{funcname paramspec resulttype . body}{
defines a EusLisp function that can be called from foreign language code.
{\em funcname} is a symbol for which a EusLisp function is defined.
{\em paramspec} is a list of type specifiers as in {\bf defforeign}.
Unlike {\bf defforeign}'s paramspec, {\bf defun-c-callable}'s paramspec
cannot be omitted unless the function does not receive any argument.
{\em :integer} should be used for all of int, short and char types
and {\em :float} for both of float and double.
{\em resulttype} is the type of the Lisp function.
{\em resulttype} can be omitted unless you need type check or type coercion from integer
to float.
{\em body} is lisp expressions that are executed when this function is
called from C.
The function defined by {\bf defun-c-callable} can be called from Lisp
expressions, too.
{\bf Defun-c-callable} returns {\em funcname}.
It looks like a symbol, but it is not, but an instance of {\bfx foreign-pod}
which is a subclass of symbol.}


\funcdesc{pod-address}{funcname}{
returns the address of a foreign-to-EusLisp interface code of the
c-callable Lisp function {\em funcname} defined by {\bf defun-c-callable}.
This is used to inform a foreign language program of the
location of a Lisp function.}

\macrodesc{array-entity}{array-of-more-than-one-dimension}{
returns one-dimensional vector which holds all the elements of
a multi-dimensional array.
This is needed to pass a multi-dimensional or general array to a foreign function,
although a simple vector can be passed directly.}

\funcdesc{float2double}{float-vector [doublevector]}{
converts a float-vector to double precision representation.
The result is of type float-vector but the length is twice as much as the
first argument.}

\funcdesc{double2float}{doublevector [float-vector]}{
A vector of double precision numbers is converted to single precision
float-vector.}

\end{refdesc}

