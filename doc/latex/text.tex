\section{Text Processing}

\subsection{Japanese Text}
Japanese characters are encoded in 16-bit, i.e. two bytes.
Inside EusLisp, there is no provision to handle Japanese 16-bit
character as a representation of Japanese.  They are just regarded
as a series of byte-encoded characters.
The following code will print a Japanese character "AI" that
means {\it love} in English, if you are using a terminal
that can display EUC kanji, like kterm.
\begin{verbatim}
(setq AI-str 
      (let ((jstr (make-string 2)))
          (setf (aref jstr 0) #xb0
	        (aref jstr 1) #xa6)
	  jstr))
(print AI-str)
\end{verbatim}

In a similar manner, (intern AI-str) will create a symbol
with its printname "AI".
\begin{verbatim}
(set (intern AI-str) "love")
\end{verbatim}

Conversion functions for different character codes and
Roman-ji representation are provided.

\begin{refdesc}
\funcdesc{romkan}{romanji-str}{Roman-ji representation is 
converted into EUC coded Japanese.
Numbers are converted into pronunciation in hiragana.}

\funcdesc{romanji}{kana-str}{kana-str which represents
Japanese in hiragana or in katakana coded in EUC
is converted into a roman-ji representation.
English alphabets and numbers are unchanged.}

\funcdesc{sjis2euc}{kana-str}{kana-str coded in shift-jis
is converted into EUC.}

\funcdesc{euc2sjis}{kana-str}{kana-str coded in EUC
is converted into shift-JIS.}

\funcdesc{jis2euc}{kana-str}{kana-str coded in EUC
is converted into JIS coding, which enters kana mode by
\verb+ESC\$B+ and exits by \verb+ESC(J+.
Note that there is no euc2jis function is provided yet.}

\funcdesc{kana-date}{time}{time is converted a Japanese
date pronunciation represented in roman-ji. The default time
is the current time.}

\funcdesc{kana-date}{time}{time is converted a Japanese
time pronunciation represented in roman-ji. The default time
is the current time.}

\funcdesc{hira2kata}{hiragana-str}{
hiragana-str is converted into katakana representation.}

\funcdesc{kata2hira}{katakana-str}{
katakana-str is converted into hiragana representation.}

\end{refdesc}

\subsection {ICONV - Character Code Conversion}

ICONV is a set of the gnu standard library functions for character
code conversion.  The interface is programmed in eus/lib/clib/charconv.c.


\begin{refdesc}
\funcdesc{iconv-open}{to-code from-code}{returns a descriptor for
converting characters from \it{from-code} to \it{to-code}.}
\end{refdesc}


\subsection{Regular Expression}

\begin{refdesc}

\funcdesc{regmatch}{regpat string}{searches for an occurence of a regular
expression, {\it regpat} in {\it string}.
If found, a list of the starting index and the ending index
of the found pattern is returned.
example;
(regmatch "ca[ad]+r" "any string ...") will look for cadr, caar, cadadr ...
 in the second argument.
}

\end{refdesc}

\subsection{Base64 encoding}

Base64 is an encoding scheme to represent binary data using only
printable graphic characters.  The scheme is applied to
uuencode/uudecode.  The following functions are defined in
lib/llib/base64.l.

\begin{refdesc}
\funcdesc{base64encode}{binstr}{
A binary string, {\it binstr } is converted to an ASCII string
consisting only of \[A-Za-z0-9+/=\] letters
according to the base-64 encoding rule.
The resulted string is 33\% longer than the original.
A newline is inserted every 76 characters.
One or two '=' characters are padded at the end to adjust the
length of the result to be a multiple of four.}

\funcdesc{base64decode}{ascstr}{
An ASCII string, {\it ascstr}, is converted to a binary string
according to the base-64 encodeing.
Error is reported if ascstr includes an invalid character.}
\end{refdesc}

\subsection{DES cryptography}
Linux and other UNIX employs the DES (Data Encryption Standard)
to encrypt password strings.
The function is provided in the libcrypt.so library.
lib/llib/crypt.l links this library and provides the following
functions for string encryption.
Note that the $2^56$ key space of DES is not large enough to
reject challenges by current powerful computers.  Note also
that only the encrypting functions are provided and no
rational decrypting is possible.

\begin{refdesc}
\funcdesc{crypt}{str salt}{
The raw function provided by libcrypt.so.
{\it Str} is encrypted by using the {\it salt} string.
{\it Salt} is a string of two characters, and used to randamize
the output of encryption in 4096 ways.
The output string is always 13 characters regardless to the
length of {\it str}.
In other words, only the first eight characters from {\it str}
are taken for encryption, and the rest are ignored.
The same string encrypted with the same salt is the same.
The same string yields different encryption result 
with different salts.
The salt becomes the first two characters of the resulted
encrypted string.}

\funcdesc{rcrypt}{str \&optional (salt (random-string 2))}{
The plain string, {\it str}, is converted into its encrypted
representation.  The {\it salt} is randomly generated if
not given.}

\funcdesc{random-string}{len \&optional random-string}{
This is a utility function to generate a random string
which constitutes of elements in the {\it random-string}.
By default, "A-Za-z0-9/." is taken for the {\it random-string}.
In order not to make mistakes between i, I, l, 1, O, 0, and o,
you can specify *safe-salt-string* for the {\it random-string}.}

\funcdesc{compcrypt}{input cryption}{
{\it Input} is a plain string and {\it cryption} is a encrypted
string.  {\it Input} is encrypted with the salt found in the {\it cryption}
and the result is compared with it.  If both are the same,
T is returnd, NIL, otherwise.}

\end{refdesc}
